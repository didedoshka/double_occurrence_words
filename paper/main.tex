\documentclass[a4paper,fleqn,12pt,top=0pt]{article}
\usepackage[russian]{babel}
\usepackage{amsmath}
\usepackage{amsthm}
\usepackage{amsfonts}
\usepackage{graphicx}
\usepackage[export]{adjustbox}
\usepackage[nottoc]{tocbibind}
\usepackage[square,numbers]{natbib}
\usepackage{svg}
\usepackage{hyperref}

\usepackage[nolists,nomarkers]{endfloat}

\usepackage[left=2cm,right=2cm, top=2cm,bottom=2cm]{geometry}
% \usepackage[top=15mm,left=15mm,right=15mm,bottom=15mm]{geometry}

\theoremstyle{plain}

\newtheorem{theorem}{Теорема}
\newtheorem{lemma}[theorem]{Лемма} 
\newtheorem{proposition}[theorem]{Предложение}
\newtheorem{corollary}[theorem]{Следствие}
\newtheorem{claim}[theorem]{Утверждение}
\newtheorem{algorithm}[theorem]{Алгоритм}

\theoremstyle{definition}
\newtheorem{definition}[theorem]{Определение}

\theoremstyle{remark}
\newtheorem{example}[theorem]{Пример}
\newtheorem{examples}[theorem]{Примеры}
\newtheorem{remark}[theorem]{Замечание}

\newcommand{\An}{\operatorname{An}}

\bibliographystyle{abbrvnat}

\title{Изучение свойств сборных графов – палиндромов и полупалиндромов}
\author{Дмитрий Горохов}

\begin{document}

\tableofcontents

\newpage

\section{Аннотация}

Проект посвящен 2-словам, которые играют важную роль в генетике при описании эпигенетических геномных перестроек. Удобным геометрическим представлением 2-слов являются так называемые сборные графы. Среди их характеристик выделяется сборное число – минимальное количество полигональных путей, покрывающих все 4-валентные вершины графа.
В данном проекте будет сделан упор на 2-слова, являющиеся палиндромами и полупалиндромами: будут исследоваться их комбинаторные свойства и характеристики. Один из интересующих вопросов таков: верно ли, что сборное число любого полупалиндрома равно 1?

\newpage
\section{Обзор источников}

Статья\cite{article1} содержит определение таких понятий, как сборный граф, изоморфизм сборных графов, трансверсаль, простой сборный граф, слово, сборное слово или 2-слово, разложимый и неразложимый сборный граф, Гамильтоново множество путей, полигональный путь, сборное число, реализуемый и нереализуемый сборный граф, минимальное реализующее число, а также некоторые теоремы.

Последующая статья по теме\cite{article2} определяет такие понятия, как 2-слово в порядке возрастания, палиндром, сильно-неразложимый сборный граф, а также выводит формулы для подсчета всех, неразложимых и сильно-неразложимых 2-слов, а также всех, неразложимых и сильно-неразложимых палиндромов.

В статье\cite{article3} рассматриваются матрицы инцедентности простых сборных графов и их свойства, серии сборных графов, широко известные благодаря их экстремальным свойствам, а также внутренняя петельная подстановка.
\newline

Рассматриваются конечные графы $\Gamma = (V, E)$, где $V$ --- множество вершин, а $E \subseteq V \times V$ --- множество ребер. Граф может содержать петли и кратные ребра.

\begin{definition}
    Степень вершины $v \in V$ --- число ребер, инцидентных данной вершине.
\end{definition}

\begin{definition}
    Циклический порядок для кортежа из $k$ элементов $(x_1, x_2, x_3, \dots, x_{k - 1}, x_k)$ --- множество 
    
    \begin{gather*}
    (x_1, x_2, x_3, \dots, x_{k - 1}, x_k)^{cyc} = \{
    (x_1, x_2, x_3, \dots, x_{k - 1}, x_k), \\
    (x_2, x_3, \dots, x_{k - 1}, x_k, x_1),
    (x_3, \dots, x_{k - 1}, x_k, x_1, x_2), \dots, \\
    (x_k, x_1, x_2, x_3, \dots, x_{k - 1}),
    (x_k, x_{k - 1}, x_{k - 2}, \dots, x_2, x_1), \\
    (x_{k - 1}, x_{k - 2}, \dots, x_2, x_1, x_k),
    (x_{k - 2}, \dots, x_2, x_1, x_k, x_{k - 1}), \dots, \\
    (x_1, x_k, x_{k - 1}, x_{k - 2}, \dots, x_2)
    \}
    \end{gather*}
    то есть все циклические сдвиги кортежа и все циклические сдвиги кортежа, записанного в обратном порядке.
    
    Чтобы задать циклический порядок, достаточно одного элемента множества $(x_1, x_2, x_3,$ $\dots, x_{k - 1}, x_k)^{cyc}$.
\end{definition}

\begin{definition}
    Вершина $v$ --- упорядоченная (или, иногда, регулярная), если циклический порядок ребер, инцидентных ей, зафиксирован.
\end{definition}

\begin{remark}
    Для каждого из ребер упорядоченной вершины корректно определены его соседи.
\end{remark}

\begin{definition}[\protect{\cite[Definition 3.1.]{article1}}]
    Сборный граф --- конечный связный граф, в котором все вершины упорядоченные и имеют степени $1$ или $4$.
\end{definition}

\begin{definition}
    Концевая вершина --- вершина степени $1$.
\end{definition}

\begin{definition}
    Порядок $\Gamma$ (обозначение $|\Gamma|)$ --- количество вершин, степени $4$ сборного графа $\Gamma$.
\end{definition}

\begin{definition}
    Тривиальный сборный граф --- сборный граф $\Gamma$, что $|\Gamma| = 0$.
\end{definition}

\begin{definition}[\protect{\cite[Definition 3.2.]{article1}}]
    Графы $\Gamma_1$ и $\Gamma_2$ изоморфны, если $|\Gamma_1| = |\Gamma_2|$ и существует изоморфизм $\phi: V_1 \rightarrow V_2$, такой, что 
    
1. для любых $u, v \in V_1$ ребро $(u, v) \in E_1$ тогда и только тогда, когда $(\phi(u), \phi(v)) \in E_2$;

2. для любой $u \in V_1$ циклический порядок ребер в $u$ совпадает с циклическим порядком их $\phi$-образов в $\phi(u)$.
\end{definition}

\begin{definition}
    Последовательность вершин и ребер $(v_0, e_1, v_1, e_2, \dots, e_n, v_n)$ --- путь в графе, если $v_i \in V$, $e_i \in E$, $e_i$ --- ребро между $v_{i - 1}$ и $v_{i}$.
\end{definition}

\begin{definition}[\protect{\cite[Definition 3.3.]{article1}}]
    Трансверсаль $(v_0, e_1, v_1, e_2, \dots, e_n, v_n)$ --- путь, что каждая вершина встречается максимум два раза, все ребра различны, а ребра $e_i$ и $e_{i + 1}$ не являются соседними в $v_i$.
\end{definition}

\begin{definition}
    Эйлеров путь в $\Gamma$ --- путь, содержащий каждое ребро ровно один раз.
\end{definition}

\begin{definition}[\protect{\cite[Definition 3.5.]{article1}}]
    Простой сборный граф --- сборный граф, содержащий эйлерову трансверсаль.
\end{definition}

\begin{definition}
    Две трансверсали эквивалентны, или если они равны, или если одна является другой в обратном порядке.
\end{definition}

\begin{lemma}[\protect{\cite[Lemma 3.6.]{article1}}]
    В простом сборном графе существует единственный класс эквивалентности трансверсалей.
\end{lemma}

\begin{lemma}[\protect{\cite[Lemma 3.7.]{article1}}]
    Два простых сборных графа $\Gamma_1$ и $\Gamma_2$ с трансверсалями $\gamma_1$ и $\gamma_2$ изоморфны если и только если существует отображение $\Phi = (\Phi_v, \Phi_e) : \Gamma_1 \rightarrow \Gamma_2$ с биекциями $\Phi_v: V_1 \rightarrow V_2$ и $\Phi_e: E_1 \rightarrow E_2$, что $\Phi(\gamma_1)$ эквивалентна $\gamma_2$.
\end{lemma}

Сборные графы естественным образом связаны со специальным классом слов.

\begin{definition}
    Сборное слово или 2-слово --- это слово в некотором алфавите $S = \{a_1, a_2, \dots\}$, что каждая буква $a_i$ либо содержится в слове ровно два раза, либо не содержится вовсе.
\end{definition}

\begin{definition}
    Порядок 2-слова $w$ (обозначение $|w|$) --- количество букв, которые встречаются в 2-слове.
\end{definition}

\begin{definition}
    Обратное слово к слову $w = a_{i_1} \dots a_{i_k}$ (обозначение $w^R$) --- $a_{i_k} \dots a_{i_1}$.
\end{definition}

\begin{definition}
    Два 2-слова эквивалентны, если после переименования некоторых букв они или совпадают, либо являются обратными друг для друга.
\end{definition}

\begin{lemma}[\protect{\cite[Lemma 3.8.]{article1}}]
    Классы эквивалентности 2-слов находятся в биективном соответствии с классами изоморфизма простых сборных графов.
\end{lemma}

\begin{definition}[\protect{\cite[Definition 3.10.]{article1}}]
    Композиция $\Gamma_1 \circ \Gamma_2$ двух ориентированных простых сборных графов $\Gamma_1$ и $\Gamma_2$ --- это граф, который получается, если отождествить конечную вершину $\Gamma_1$ и начальную вершину $\Gamma_2$, после чего забыть об этой вершине.
\end{definition}

\begin{remark}
    Композиция простых сборных графов --- простой сборный граф.
\end{remark}

\begin{definition}[\protect{\cite[Definition 3.11.]{article1}}]
    Разложимое 2-слово $w$ --- такое 2-слово, которое может быть записано как произведение $w = uv$ двух непустых 2-слов $u, v$. Аналогично, разложимый простой сборный граф $\Gamma$ --- такой сборный граф, что $\Gamma = \Gamma_1 \circ \Gamma_2$ для непустых простых сборных графов $\Gamma_1, \Gamma_2$. В противном случае и 2-слово, и простой сборный граф --- неразложимые.
\end{definition}

Далее рассматриваются полигональные пути и сборное число.

\begin{definition}
    Простой путь --- путь, не содержащий какую-либо вершину дважды.
\end{definition}

\begin{definition}[\protect{\cite[Definition 4.2.]{article1}}]
    Множество попарно непересекающихся простых путей $\{\gamma_1, \dots, \gamma_k\}$ --- гамильтоново, если их объединение содержит все вершины степени $4$ графа $\Gamma$.
\end{definition}

\begin{definition}[\protect{\cite[Definition 4.3.]{article1}}]
    Полигональный путь --- путь $(v_0$, $e_1, v_1, e_2 ,\dots, e_n, v_n)$, что $e_i$  $e_{i + 1}$ --- соседи для $v_i$ для $i \in \{1, \dots, n-1\}$.
\end{definition}

\begin{definition}[\protect{\cite[Definition 4.4.]{article1}}]
    Сборное число простого сборного графа $\Gamma$ (обозначение $\An(\Gamma)$), определяется как $\An(\Gamma) = \min\{k | $ существует гамильтоново множество полигональных путей $\{\gamma_1, \dots, \gamma_k\}$ в $\Gamma\}$.
\end{definition}

\begin{definition}
    Реализумый простой сборный граф --- простой сборный граф, со сборным числом $1$. Иначе --- нереализуемый.
\end{definition}

\begin{lemma}[\protect{\cite[Lemma 4.6.]{article1}}]
    Для любой пары ориентированных простых сборных графов $\Gamma_1$ и $\Gamma_2$, одно из двух равенств выполнено: $\An(\Gamma_1 \circ \Gamma_2) = \An(\Gamma_1) + \An(\Gamma_2)$, или $\An(\Gamma_1 \circ \Gamma_2) = \An(\Gamma_1) + \An(\Gamma_2) - 1$.
\end{lemma}

\begin{proposition}[\protect{\cite[Proposition 4.9.]{article1}}]
    Для любого натурального числа $n$
    \begin{enumerate}
        \item Существует разложимый сборный граф $\Gamma$, что $\An(\Gamma) = n$.
        \item Существует неразложимый сборный граф $\Gamma$, что $\An(\Gamma) = n$.
    \end{enumerate}
\end{proposition}

\begin{definition}[\protect{\cite[Definition 4.10.]{article1}}]\label{rmin}
    Минимальное реализующее число для натурального числа $n$ (обозначение $R_{\min}(n)$) определяется как $R_{\min}(n) = \min\{|\Gamma|:\An(\Gamma) = n\}$. Граф $\Gamma$, такой что $R_{\min}(n) = |\Gamma|$, --- реализация $R_{\min}(n)$.
\end{definition}

\begin{proposition}[\protect{\cite[Proposition 4.13.]{article1}}]
    Следующие свойства выполняются для $R_{\min}$
    \begin{enumerate}
        \item Для любого натурального числа $n$ $R_{\min}(n) < R_{\min}(n + 1)$.
        \item Если $R_{\min}(n) = k$, то $\forall s \geq k $ существует сборный граф $\Gamma$, что $|\Gamma| = s, \An(\Gamma) = n$.
        \item Для любого натурального числа $n$ $R_{\min}(n) \leq 3(n - 1) + 1$.
    \end{enumerate}
\end{proposition}

\begin{proposition}[\protect{\cite[Proposition 4.14.]{article1}}]
    Существует константа $N$ такая, что для любого реализуемого 2-слова $w$, существует нереализуемое неразложимое $v$, что $w \subset v$ и $|v| - |w| < N$.
\end{proposition}

\begin{proposition}[\protect{\cite[Proposition 4.17.]{article1}}]
    Для любого нереализуемого 2-слова $v$, что $|v| = m$, существует константа $N(m)$ и реализуемое 2-слово $w$, что $v \subset w$ и $|w| - |v| \leq N(m)$.
\end{proposition}

\begin{definition}
    Пусть слова записаны над алфавитом линейно сравнимых элементов. Тогда говорят, что слово записано в порядке возрастания, если $i$-ая буква в слове по величине встречается в слове первый раз только после всех букв, меньших ее.
\end{definition}

Далее мы отождествляем 2-слово и его запись в возрастающем порядке.

Далее рассматриваются комбинаторные свойства 2-слов.

\begin{lemma}[\protect{\cite[Lemma 3.2.]{article2}}]
    Мощность множества 2-слов на $n$ буквах есть 
    \[
    W_n = (2n - 1)!!
    \]
\end{lemma}

\begin{definition}[\protect{\cite[Definition 3.3.]{article2}}]
    Палиндром --- такое 2-слово, что его обратное (записанное в возрастающем порядке) равно ему.
\end{definition}

\begin{lemma}[\protect{\cite[Lemma 3.4.]{article2}}]
    Количество палиндромов на $n$ буквах есть
    \[
    P_n = \sum_{k=\lfloor n/2 \rfloor}^{n} \binom{k}{n - k} \frac{n!}{k!}
    \]
\end{lemma}

\begin{lemma}[\protect{\cite[Lemma 3.6.]{article2}}]
    Количество неразложимых 2-слов на $n$ буквах есть 
    \[
    I_1 = 1; \\
    I_n = W_n - \sum_{k = 1}^{n - 1}W_kI_{n - k}
    \]
\end{lemma}

\begin{lemma}[\protect{\cite[Lemma 3.7.]{article2}}]
    Количество неразложимых палиндромов на $n$ буквах есть
    \[
    J_1 = 1; \\ J_n = P_n - \sum_{k = 1}^{\lfloor n / 2 \rfloor}W_kJ_{n - 2k}
    \]
\end{lemma}

\begin{definition}[\protect{\cite[Definition 3.8.]{article2}}]
    Сильно-неразложимое 2-слово --- такое 2-слово, что оно не содержит никакого собственного 2-подслова.
\end{definition}

\begin{lemma}[\protect{\cite[Proposition 3.10]{article2}}]
    Количество сильно-неразложимых 2-слов на $n$ буквах есть 
    \[
    S_1 = 1; \\
    S_n = (n - 1) \sum_{i = 1}^{n - 1}S_i S_{n - i}
    \]
\end{lemma}

\begin{lemma}[\protect{\cite[Proposition 3.10]{article2}}]
    Количество сильно-неразложимых палиндромов на $n$ буквах есть 
    \[
    T_0 = -1; \\
    T_1 = 1; \\
    T_n = (n - 1) \sum_{i = 1}^{n - 2}T_i T_{n - i} + \sum_{i = 1}^{\lfloor n / 2 \rfloor} (2n - 4i - 1)S_i T_{n - 2i}
    \]
\end{lemma}

\newpage
\section{Полученные результаты}
\subsection{Теоретические}

\begin{definition}
    2-слово $w$ на $n$ буквах в возрастающем порядке --- полупалиндром, если $\forall i \in \{1, \dots, 2n\}:  w_{2n - i + 1} = n - w_i + 1$.
\end{definition}

\begin{example}
    1122 и 1212 --- полупалиндромы, а 1221 --- нет.
\end{example}

\begin{definition}
    Скобочная последовательность --- символьная последовательность, состоящая из символов ``('' и ``)''
\end{definition}

\begin{definition}[\protect{\cite[2.10.1 Правильные последовательности скобок]{hse}}]
    Правильная скобочная последовательность (ПСП) определяется индуктивно
    \begin{itemize}
        \item пустая последовательность правильна;
        \item если $A$ --- правильная скобочная последовательность, то $(A)$ --- правильная скобочная последовательность;
        \item если $A, B$ --- правильные скобочные последовательности, то их конкатенация $AB$ --- правильная скобочная последовательность.
    \end{itemize}
\end{definition}

\begin{definition}
    Префикс ПСП длины $n$ --- скобочная последовательность, которую можно дополнить до правильной.
\end{definition}

\begin{definition}
    Баланс скобочной последовательности --- разность количества открывающихся скобок и количества закрывающихся.
\end{definition}

\begin{lemma}
    Аналогично \cite[Теорема 2.2.]{hse} скобочная последовательность является префиксом ПСП тогда и только тогда, когда на любом ее префиксе баланс неотрицателен.
\end{lemma}

\begin{definition}
    Симметричная ПСП --- такая ПСП, что если на $i$ месте стоит открывающаяся скобка, то на $n - i + 1$ стоит закрывающаяся, а если закрывающаяся, то открывающаяся.
\end{definition}

\begin{remark}
    Можно провести естественное отображение между 2-словами в возрастающем порядке и ПСП: первому вхождению символа сопоставить открывающуюся скобку, второму вхождению --- закрывающуюся. Это отображение не является инъекцией: 1212 и 1221 переходят в (()). При этом это отображение --- сюръекция.
\end{remark}

\begin{proposition}\label{semi_bijective_cps}
    Множество полупалиндромов на $n$ буквах биективно множеству симметричных ПСП длины $2n$.
\end{proposition}

\begin{proof}
    Рассмотрим отображение $f$ из множества симметричных ПСП в множество полупалиндромов: отдельно пронумеруем открывающиеся скобки в порядке возрастания, отдельно закрывающиеся в порядке возрастания и запишем это в строку. Примеры: $()()() \rightarrow 112233$, $(()()) \rightarrow 121323$,$((())) \rightarrow 123123$. Заметим, что скобки, которые при анализе ПСП разбиваются на пары ``открывающая-закрывающая'' не соответствуют парам букв.
    
    Покажем, что оно действительно бьет в множество полупалиндромов. Для начала, результат отображения --- 2-слово, так как количество открывающихся и закрывающихся скобочек одинаково. Результат в возрастающем порядке --- первый раз каждое число встречается в том месте, в котором в ПСП стоит открывающаяся скобка (так как закрывающаяся скобка номер $k$ всегда идет после открывающейся скобки номер $k$), а значит первое вхождение числа $k$ будет позже, чем первые вхождений чисел $1 \dots (k-1)$. Результат --- полупалиндром в силу симметричности ПСП: если на $i$-ом ($i \le n$) месте стоит $k$-ая открывающаяся скобочка, то на $n - i + 1$ будет стоять закрывающаяся, и ее номер будет равен $n - k + 1$, и аналогично с закрывающейся.

    $f$ --- инъекция: для любых двух различных ПСП в месте их первого отличия в образе будут стоять разные числа.

    Покажем, что $f$ --- сюръекция. Рассмотрим полупалиндром $w$ на $n$ буквах. Построим следующую скобочную последовательность $s$: если $w_i$ встречается в $w$ первый раз, $s_i = ($, иначе $s_i = )$. $s_i$ --- ПСП, так как баланс равен нулю и на любом префиксе баланс неотрицателен (так как каждое число мы встретим первый раз хотя бы столько же раз, сколько второй раз). $s_i$ --- симметричная ПСП, так как если в полупалиндроме $w_i$ встречается первый раз, то $w_{n - i + 1}$ --- второй, и наоборот. Так как  $f(s) = w$ получаем, что $f$ --- сюръекция.

    Так как $f$ --- инъекция и сюръекция, то $f$ --- биекция.
\end{proof}

Этой биекции можно придать больше смысла, введя для полупалиндрома аналогию открывающихся и закрывающихся скобок.
\begin{definition}
    Пусть $p$ --- полупалиндром порядка $n$. Тогда $s$ --- префикс полупалиндрома $p$, если $s$ является префиксом $p$ как строки и длина $s$ меньше $n$.
\end{definition}
\begin{proposition}\label{meaning}
    Пусть $s$ --- префикс полупалиндрома. Тогда $s$ можно продолжить 
    \begin{itemize}
        \item единственным способом, если каждая буква встречается в $s$ дважды. Этот способ --- новая буква (следующая по возрастанию после наибольшей среди встречающихся). Эта ситуация соответствует префиксу ПСП с нулевым балансом, единственный способ продолжить который --- открывающая скобка.
        % \item единственным способом, если баланс соответствующего префикса ПСП равен нулю. Данный способ соответствует появлению новой буквы в слове.
        \item двумя способами, если существует буква, которая встречается в $s$ один раз. Первый способ --- новая буква (следующая по возрастанию после наибольшей среди встречающихся). Второй способ --- наименьшая буква префикса, которая встречалась один раз. Эта ситуация соответствует префиксу ПСП с ненулевым балансом, первый способ продолжить который --- открывающая скобка, второй --- закрывающая.
        % \item двумя способами, если баланс ненулевой. В таком случае может либо появиться новая буква (открывающая скобка) либо --- второе вхождение наименьшей буквы префикса, которая встречалась один раз (закрывающая скобка).
    \end{itemize}

    % Имея какой-то префикс полупалиндрома (длины меньше, чем $n$), его можно продолжить одним или двумя способами (аналогично тому, как префикс ПСП можно продолжить одним способом, если баланс равен нулю (добавить открывающуюся скобку), и двумя иначе (или открывающуюся, или закрывающуюся)). Префикс полупалиндрома можно продолжить минимальным числом, которое еще не встречалось, или, если такое есть, минимальным числом, которое встречалось один раз.
\end{proposition}

\begin{example}
    112 можно продолжить или 2, или 3, а 1212 только 3. При этом, 123 нельзя продолжить ни 2, ни 3, а только 1 или 4.
\end{example}

\begin{proof}
    Очевидно, мы не можем продолжить числом, большим чем минимальное, которое еще не встречалось --- таким образом, мы нарушим возрастающий порядок. Покажем, что второе вхождение числа $m$, $m>k$ в полупалиндроме невозможно раньше, чем второе вхождение числа $k$. Пусть $k$ --- минимальное число, которое встречается один раз до того момента, как какое-то $m$ встречается два раза. У нас может быть два случая: $k$ второй раз встретился в первой половине полупалиндрома, или во второй. В первом случае у нас нарушится возрастающий порядок во второй половине полупалиндрома: в первой у нас стоит $k...m.m..k..$, значит во второй будет стоять $..(n-k+1)..(n-m+1).(n-m+1)...(n-k+1)$, но $n - k + 1$ больше, чем $n - m + 1$. Во втором случае $k$ второй раз встречается во второй половине полупалиндрома, значит $n - k + 1$ встречается в первой. Но это означает, что и все числа меньше $n - k + 1$ тоже встречаются в первой, то есть там встречается $n - k + 1$ различное число. При этом $k$ --- первое число, которое встречается один раз, значит все числа от $1$ до $k-1$ встречаются два раза, а еще $m$ встречается два раза. В $n$ мест не помещается $n - k + 1 + k - 1 + 1 = n + 1$ число. Получается противоречие, доказывающее требуемое.
\end{proof}

\begin{lemma}
    Множество префиксов ПСП длины $n$ биективно множеству симметричных ПСП длины $2n$.
\end{lemma}
\begin{proof}
    Пусть $p$ --- префикс ПСП. Если его отразить (развернуть и заменить открывающиеся скобки на закрывающиеся и наоборот) и присоединить к $p$, получится симметричная ПСП. 
\end{proof}

\begin{proposition}
    Количество полупалиндромов на $n$ буквах есть $SP_n = \binom{n}{\lfloor \frac{n}{2} \rfloor}$.
\end{proposition}
\begin{proof}
    Количество полупалиндромов на $n$ буквах равно количеству префиксов ПСП длины $n$ по лемме \ref{semi_bijective_cps}.

    Пусть $P_n^k$ --- количество префиксов ПСП длины $n$, в которых меньше либо равно $k$ закрывающихся скобок. Если $k > \lfloor \frac{n}{2} \rfloor$, то $P_n^k = P_n^{k-1}$. Действительно, ведь существует ноль префиксов ПСП, в которых больше половины закрывающихся скобок. Иначе, если $k \le \lfloor \frac{n}{2} \rfloor$, то $P_n^k = P_{n-1}^k + P_{n-1}^{k-1}$, так как последний символ может быть или открывающейся скобкой, или закрывающейся. Из-за рекуррентной формулы и начальных условий замечаем, что $P_n^{\lfloor \frac{n}{2} \rfloor} = \binom{n}{\lfloor \frac{n}{2} \rfloor}$, а $P_n^{\lfloor \frac{n}{2} \rfloor}$ --- количество всех префиксов ПСП длины $n$.
    
    Данная последовательность содержится в On-Line Encyclopedia of Integer Sequences\cite{oeis} с индексом A001405.
\end{proof}

\begin{proposition}
    Количество неразложимых полупалиндромов на $n$ буквах равно количеству сильно-неразложимых полупалиндромов на $n$ буквах.
\end{proposition}
\begin{proof}
    В общем случае верно, что сильно-неразложимое 2-слово является неразложимым 2-словом. Надо показать, что неразложимый полупалиндром является сильно-неразложимым полупалиндромом.
    
    Для того, чтобы это показать, покажем, что если у полупалиндрома есть 2-подслово, то у него есть и префикс, являющийся 2-словом. Обозначим данное 2-подслово как $w$. Если $w$ префикс, то мы нашли искомый префикс. Если $w$ находится целиком в первой половине, то рассмотрим $u$ --- префикс до начала $w$ невключительно. Пакажем, что $u$ --- 2-слово. Пусть это не так, то есть какая-то буква $a$ встречается в нем только один раз. Если $a$ не встречается второй раз в $u$, значит второй раз она встречается после конца $w$ (она не может встречаться в $w$, так как в нем каждая буква встречается два раза). Но $w_1 > a$, при этом встречается второй раз раньше, чем $a$ встречается второй раз, что противоречит предложению \ref{meaning}, а значит $u$ --- 2-слово. Если $w$ лежит целиком во второй половине, то симметрично ему есть 2-слово, которое целиком лежит в первой половине, из-за симметричности полупалиндрома. Остается случай, если $w$ пересекает середину. Тут есть два случая. Первый случай, если середина $w$ не является серединой полупалиндрома. Тогда симметрично $w$ тоже лежит 2-слово $v$. Так как середина $w$ не является серединой полупалиндрома, в $w$ есть часть, которая не лежит в $v$. Эта часть является 2-словом, так как та часть, которая лежит и в $w$, и в $v$ является 2-словом (потому что все буквы, которые в ней находятся, находятся оба раза и в $w$, и в $v$). Эта часть лежит целиком или в первой половине, или во второй, значит мы перешли к уже рассмотренному случаю. Последний случай --- середина $w$ совпадает с серединой полупалиндрома. $w$ разделяется серединой на $w_1$ и $w_2$. Если в $w_1$ какая-то буква встречается два раза, значит префикс до $w$ невключительно является 2-словом аналогично случаю, разобранному ранее. Если в $w_1$ каждая буква встречается по одному разу, значит $w_1$ длины $m$ есть $k(k+1)\dots(k+m-1)$. Так как полупалиндром, $w_2$ тоже возрастает, и чтобы $w$ было 2-словом, надо, чтобы $n - (k + m - 1) + 1 = k$, то есть $n - m = 2(k - 1)$. До буквы $k$ встречаются все $k - 1$ меньшие буквы. Они стоят на $n - m$ местах (на оставшихся $m$ местах первой половины стоит $w_1$). То есть $k - 1$ букв стоят на $2(k - 1)$ местах, значит каждая буква стоит два раза, значит это 2-слово.
\end{proof}

\begin{proposition}
    Количество сильно-неразложимых (просто неразложимых) полупалиндромов на $n$ буквах есть $SPI_n = \binom{n-1}{\lfloor \frac{n-1}{2} \rfloor}$.
\end{proposition}
\begin{proof}
    Если есть префикс, являющийся 2-словом, длины больше $n$, то оставшийся суффикс тоже является 2-словом, и в силу симметричности полупалиндрома это значит, что есть и префикс, являющийся 2-словом, длины меньше либо равной $n$. 

    Рассмотрим биекцию между полупалиндромами и префиксами ПСП. Заметим, что если у полупалиндрома есть префикс длины $k$, являющийся 2-словом, то при биекции он перейдет в префикс ПСП, у которого баланс префикса длины $k$ есть ноль. Получается неразложимые полупалиндромы переходят в префиксы ПСП, у которых никогда нет баланса ноль. То есть в скобочные последовательности, баланс на любом префиксе которых больше либо равен единице. Такие скобочные последовательности длины $n$ --- открывающаяся скобка конкатенированная с префиксом ПСП длины $n - 1$, ведь префиксы ПСП --- скобочные последовательности, баланс на любом префиксе которых больше либо равен нуля. Получается количество искомых объектов длины $n$ равно количеству префиксов ПСП длины $n - 1$, то есть $\binom{n-1}{\lfloor \frac{n-1}{2} \rfloor}$.
\end{proof}

\begin{table}
    \centering
    \begin{tabular}{|l|l|l|l|}
\hline
$n$ & Все слова $W_n$ & Палиндромы $P_n$ & Полупалиндромы $SP_n$ \\
\hline
1 & 1 & 1 & 1 \\
2 & 3 & 3 & 2 \\
3 & 15 & 7 & 3 \\
4 & 105 & 25 & 6 \\
5 & 945 & 81 & 10 \\
6 & 10395 & 331 & 20 \\
7 & 135135 & 1303 & 35 \\
8 & 2027025 & 5937 & 70 \\
9 & 34459425 & 26785 & 126 \\
10 & 654729075 & 133651 & 252 \\
11 & 13749310575 & 669351 & 462 \\
12 & 316234143225 & 3609673 & 924 \\
13 & 7905853580625 & 19674097 & 1716 \\
14 & 213458046676875 & 113525595 & 3432 \\
15 & 6190283353629375 & 664400311 & 6435 \\
\hline
OEIS\cite{oeis} & A001147 & A047974 & A001405 \\
\hline
    \end{tabular}
    \caption{Количество 2-слов, палиндромов и полупалиндромов.}
\end{table}

\begin{table}
    \centering
    \begin{tabular}{|l|l|l|l|}
\hline
$n$ & Неразл. $I_n$ & Неразл. палин. $J_n$ & Неразл. полупалин. $SPI_n$ \\
\hline
1 & 1 & 1 & 1 \\
2 & 2 & 2 & 1 \\
3 & 10 & 6 & 2 \\
4 & 74 & 20 & 3 \\
5 & 706 & 72 & 6 \\
6 & 8162 & 290 & 10 \\
7 & 110410 & 1198 & 20 \\
8 & 1708394 & 5452 & 35 \\
9 & 29752066 & 25176 & 70 \\
10 & 576037442 & 125874 & 126 \\
11 & 12277827850 & 637926 & 252 \\
12 & 285764591114 & 3448708 & 462 \\
13 & 7213364729026 & 18919048 & 924 \\
14 & 196316804255522 & 109412210 & 1716 \\
15 & 5731249477826890 & 642798510 & 3432 \\
\hline
OEIS\cite{oeis} & A000698 & A195186 & A001405 \\
\hline
    \end{tabular}
    \caption{Количество неразложимых 2-слов, неразложимых палиндромов и неразложимых полупалиндромов.}
\end{table}

\begin{table}
    \centering
    \begin{tabular}{|l|l|l|l|}
\hline
$n$ & Сил.-нер. $S_n$ & Сил.-нер. пал. $T_n$ & Сил.-нер. полупал. $SPI_n$ \\
\hline
1 & 1 & 1 & 1 \\
2 & 1 & 1 & 1 \\
3 & 4 & 2 & 2 \\
4 & 27 & 7 & 3 \\
5 & 248 & 22 & 6 \\
6 & 2830 & 96 & 10 \\
7 & 38232 & 380 & 20 \\
8 & 593859 & 1853 & 35 \\
9 & 10401712 & 8510 & 70 \\
10 & 202601898 & 44940 & 126 \\
11 & 4342263000 & 229836 & 252 \\
12 & 101551822350 & 1296410 & 462 \\
13 & 2573779506192 & 7211116 & 924 \\
14 & 70282204726396 & 43096912 & 1716 \\
15 & 2057490936366320 & 256874200 & 3432 \\
\hline
OEIS\cite{oeis} & A000699 & A004300 & A001405 \\
\hline
    \end{tabular}
    \caption{Количество сильно-неразложимых 2-слов, сильно-неразложимых палиндромов и сильно-неразложимых полупалиндромов.}
\end{table}

Далее речь пойдет о гипотезе, что все полупалиндромы реализуемы.

\begin{proposition}
    Не у всех полупалиндромов сборное число равно единице.
\end{proposition}
\begin{proof}
    С помощью программы \cite[\texttt{assembly\_number\_2.cpp}]{github} был найден полупалиндром 112342534566 (рисунок \ref{an_2}), который имеет сборное число $2$ и является самым коротким, наравне с 112345234566, полупалиндромом со сборным числом 2.
\end{proof}

\begin{figure}
    \centering
    \includesvg[width=0.35\linewidth]{an_2}
    \caption{Результат работы программы.}
    \label{an_2}
\end{figure}

\begin{definition}
    Аналогично $R_{min}$ (определение \ref{rmin}) определим $R_{min}^P$ и $R_{min}^{SP}$ --- минимальное реализующее число для палиндромов и полупалиндромов.
\end{definition}

\begin{proposition}
    В предположении, что $R_{min}(n) = 3n - 2$, $R_{min}^P$ совпадает с $R_{min}$.
\end{proposition}
\begin{proof}
    Серия примеров для $R_{min}$ состоит из палиндромов, поэтому является серией примеров и для $R_{min}^P$.
\end{proof}

\begin{figure}
    \centering
    \includesvg[width=0.55\linewidth]{an_4}
    \caption{Пример полупалиндрома со сборным числом 4.}
    \label{an_4}
\end{figure}

Рассмотрим серию 2-слов $\{K_i\}$. $K_1$ ---  $11$, $K_2$ --- $112342534566$ (рисунок \ref{an_2}). $K_n$ есть $K_{n - 1}$, в конец которого добавлено (пусть $m = 5(n - 2)$) $m + 2, m + 3, m + 4, m + 2, m + 5, m + 3, m + 4, m + 5, m + 6, m + 6$. На рисунке \ref{an_4} --- $K_4$

\begin{proposition}\label{k_sp}
    $K_n$ --- полупалиндром.
\end{proposition}
\begin{proof}
    Доказательство будем вести по индукции.

    База: $K_1$ и $K_2$ являются полупалиндромами.
    
    Переход: Пусть для всех $i < n$ $K_i$ является полупалиндромом. Тогда $K_n$ --- тоже полупалиндром. 

    $n > 2$. В начале $K_n$ записано $1123425345$, в конце $m + 2, m + 3, m + 4, m + 2, m + 5, m + 3, m + 4, m + 5, m + 6, m + 6$. Заметим, что для $1 \le j \le 10$ $K_n[i] + K_n[2(5(n - 1) + 1) - i + 1] = m + 7 = 5(n - 2) + 7 = (5(n - 1) + 1) + 1$, то есть $|K_n| + 1$, что и нужно для полупалиндрома. Заметим, что то, что остается между, есть $K_{n - 2}$, но начинающийся не с единицы, а с $6$. Так как по предположению индукции $K_{n - 2}$ --- полупалиндром, то и $K_n$ --- полупалиндром.
\end{proof}

\begin{proposition}\label{k_an}
    $K_n$ имеет сборное число, равное $n$.
\end{proposition}
\begin{proof}
    Доказательство будем вести по индукции.

    База: $\An(K_1) = 1$ и $\An(K_2) = 2$.

    Переход: Пусть для все $i < n$ $\An(K_i) = i$. Тогда $\An(K_n) = n$.

    $\An(K_{n - 1}) = n - 1$. 
    
    Покажем, что $\An(K_{n}) \le n$. Действительно, гамильтоного множество полигональных путей, покрывающее весь граф, такого: возьмем минимальное гамильтоного множество полигональных путей для $K_{n - 1}$, в нем $n - 1$ элемент и добавим туда следующий путь $(m + 6, m + 5, m + 2, m + 3, m + 4)$ (рисунок \ref{path_for_proof}). Этот путь покрывает все вершины и является полигональным, а так же он покрывает только добавленные вершины, поэтому никак не пересекается с путями, которые уже были в множестве.

    Покажем, что $\An(K_{n}) \ge n$. От противного, пусть это не так, то есть пусть $\An(K_{n}) \le n - 1$. 6 максимальных вершин образуют граф $K_2$, про который мы знаем, что $\An(K_2) = 2$. При этом существует максимум один путь, который покрывает и вершины из $K_{n - 1}$, и добавленные, потому что единственное ребро, которое связывает эти два множества вершин есть ребро $(m + 1, m + 2)$. Значит если мы выкинем добавленные вершины, мы точно выкинем хотя бы один путь. Значит $\An(K_{n - 1})$ будет меньше либо равен $n - 2$, что противоречит предположению индукции и доказывает переход.

\begin{figure}
    \centering
    \includesvg[width=0.85\linewidth]{path_for_proof}
    \caption{Полигональный путь.}
    \label{path_for_proof}
\end{figure}

\end{proof}

\begin{proposition}
    Для любого натурального $n$ $R_{min}^{SP}(n) \le 5(n-1) + 1$.
\end{proposition}
\begin{proof}
    $K_n$ содержит $5(n-1)+1$ вершин, является полупалиндромом (по предложению \ref{k_sp}) и имеет сборное число, равное $n$ (по предложению \ref{k_an}).
\end{proof}

\begin{proposition}
    С помощью программы\cite{github} получена таблица \ref{an}: для каждого $n$ найдено максимальное сборное число, достижимое на полупалиндромах на $n$ буквах.
\end{proposition}
Значения в таблице, вопреки ожиданиям, невозрастают: для $n = 6$ значение равно $2$, а для $n = 7$ значение равно $1$. Это место было дополнительно проверено вручную перебором 35 графов.


\begin{table}
    \centering
    \begin{tabular}{ |l|c| } 
     \hline
     $n$ & Макс. $\An$, достиж. на полупал. на $n$ буквах \\
     \hline
     $1$ & $1$ \\
     $2$ & $1$ \\
     $3$ & $1$ \\
     $4$ & $1$ \\
     $5$ & $1$ \\
     $6$ & $2$ \\
     $7$ & $1$ \\
     $8$ & $2$ \\
     $9$ & $2$ \\
     $10$ & $2$ \\
     $11$ & $3$ \\
     $12$ & $3$ \\
     $13$ & $3$ \\
     $14$ & $3$ \\
     $15$ & $3$ \\
     \hline
    \end{tabular}
    \caption{Максимальное сборное число, достижимое на полупалиндромах на $n$ буквах.}
    \label{an}
\end{table}

\newpage
\subsection{Практические}

Была разработана библиотека\cite{github} на языке C++\cite{cpp} для работы с 2-словами и представлении их в виде сборных графов. Язык C++ был выбран из-за его быстродействия. Было решено не использовать объектно-ориентированное программирование, а избрать дизайн, ориентированный на данные\cite{data}. Это позволяет библиотеке быть легко расширяемой и способствует производительности. Быстродействие так важно, потому что алгоритм поиска сборного числа [\ref{algo}] работает экспоненциально долго. 

Библиотека реализует функции \texttt{is\_double\_occurrence\_word, \\ to\_ascending\_order, is\_in\_ascending\_order, reverse, is\_palindrome, equal\_as\_double\_occurrence\_words, is\_semi\_palindrome, is\_reducible, is\_strongly\_reducible, next\_in\_ascending\_order, next\_palindrome, next\_semi\_palindrome, assembly\_number, \\ minimal\_realization\_number\_and\_its\_realization}, функциональность которых следует из названий. 

Реализована функция \texttt{draw\_as\_graph}, использующая систему для визуализации графов Graphviz\cite{graphviz}, которая изображает 2-слово в виде сборного графа (рисунок \ref{an_2}). На рисунке соблюден порядок ребер в каждой вершине, проходя каждый раз прямо, можно пройти по трансверсали. Синим обозначено минимальное гамильтоново множество полигональных путей.

\begin{algorithm}\label{algo}
    Алгоритм для поиска сборного числа.
\end{algorithm}

Для поиска сборного числа необходимо перебрать все множества ребер, что все ребра в множестве попарно полигональны, проверить, что ребра не образуют циклов, вычислить количество путей, образованных ребрами, и вычислить минимум таких количеств по всем множествам. Пара ребер не полигональна тогда и только тогда, когда ребра этой пары идут в трансверсали не подряд. Таким образом, чтобы перебрать все множества ребер такие, что все ребра в множестве попарно полигональны, можно перебрать все множества ребер такие, что никакие два ребра не идут подряд. Опишем, как для фиксированного множества ребер $E$ проверить, что ребра из $E$ не образуют циклов и вычислить количество путей, образованных ребрами $E$. Структура ``система непересекающихся множеств''\cite{dsu} (СНМ) позволяет объединять два непересекающихся множества и проверять, лежат ли два элемента в одном множестве за $\mathcal{O}(\alpha(n))$, где $\alpha(n)$ --- обратная функция Аккермана. Заведем такую структуру для одноэлементных множеств от $\{1\}$ до $\{n\}$, где множеству $\{i\}$ сопоставляется $i$-ая вершина. Будем перебирать ребра по порядку. Если очередное ребро соединяет вершины из разных множеств, объединим их. Иначе, оно соединяет вершины из одного множества, значит ребра $E$ на вершинах этого множества образуют цикл, и множество $E$ не подходит. По окончанию этого процесса каждое множество в СНМ будет связано ребрами $E$. Так как в множестве $E$ нет пары неполигональных ребер, каждой вершине инцидентно или одно, или два ребра из $E$. Так как множество $E$ не образует циклов, получается, что каждое множество в СНМ покрыто путем из $E$. Таким образом, чтобы узнать количество путей, образованных ребрами $E$, можно посчитать количество различных множеств в СНМ. 

Пусть $n$ --- порядок $\Gamma$. Интересующих нас множеств ребер $\mathcal{O}(2^{2n})$, для каждого множества необходимо выполнить $\mathcal{O}(n \alpha(n))$ операций. Таким образом, итоговая асимптотика алгоритма есть $\mathcal{O}(2^{2n} n \alpha(n))$.

\newpage
\begin{thebibliography}{00}
	\bibitem{article1}
	A. Angeleska, N. Jonoska, M. Saito DNA recombinations through assembly graphs // Discrete Applied Mathematics. - 2009. - №157. - С. 3020-3037.

	\bibitem{article2}
	J. Burns, E. Dolzhenko, N. Jonoska, T. Muche, M. Saito Four-regular graphs with rigid vertices associated to DNA recombination // Discrete Applied Mathematics. - 2013. - №161. - С. 1378-1394.

    \bibitem{article3}
    А. Э. Гутерман, Е. М. Крейнес, Н. В. Остроухова 2-слова: их графы и матрицы // Записки научных семинаров ПОМИ. - 2019. - №482. - С. 45-72.

    \bibitem{hse}
    М.Вялый, В.Подольский, А.Рубцов, Д.Шварц, А.Шень Лекции по дискретной математике. - Изд. Дом ВШЭ, 2021. - 495 с.

    \bibitem{data}
    R. Fabian Data-oriented design. - 2018. - 307 с.

    \bibitem{dsu}
    Томас Кормен и др. Алгоритмы: построение и анализ. — 2-е изд. — М.: «Вильямс», 2006. — 1296 с.

    \bibitem{cpp}
    B. Stroustrup The C++ Programming Language. - 4-е изд. - Addison-Wesley Professional, 2013. - 1376 с.

    \bibitem{graphviz}
    Emden R. Ganser, Stephen C. North An open graph visualization system and its applications to software engineering // Software: practice and experience. - 2000. - С. 1203-1233.

    \bibitem{oeis} The On-Line Encyclopedia of Integer Sequences \url{https://oeis.org}

    \bibitem{github} Репозиторий с исходным кодом \url{https://github.com/didedoshka/double_occurrence_words}

\end{thebibliography}

\end{document}